\documentclass[aspectratio=169]{beamer}

% \usepackage{lua-visual-debug}
\usetheme{brutalb}

\title{The \texttt{BrutalB} beamer theme}
\date[ISPN ’80]{27th International Symposium of Prime Numbers}
\author[Euclid]{Euclid of Alexandria \texttt{euclid@alexandria.edu}}

\usepackage{lipsum}
\usepackage{tikz}

\begin{document}

\begin{frame}
\titlepage
\end{frame}

\section{Section Page}

\begin{frame} 
\frametitle{Fuzzy Modeling} 
\framesubtitle{The proof uses \textit{reductio ad absurdum}.} 
\begin{theorem}
There is no largest prime number. \end{theorem} 
\begin{enumerate} 
\item<1-| alert@1> Suppose $p$ were the largest prime number. 
\item<2-> Let $q$ be the product of the first $p$ numbers. 
\item<3-> Then $q+1$ is not divisible by any of them. 
\item<1-> But $q + 1$ is greater than $1$, thus divisible by some prime
number not in the first $p$ numbers.
\end{enumerate}
\end{frame}

\begin{frame}{A Tikz figure}
  \begin{figure}
    \centering
    \begin{tikzpicture}
      \pgfdeclarelayer{background layer}
      \pgfdeclarelayer{foreground layer}
      \pgfsetlayers{background layer,main,foreground layer}

      \begin{pgfonlayer}{foreground layer}
        \node[draw, align=center, fill=col1] (n1) at (-5,0) {Fuzzy Set};
        \node[draw, align=center, fill=col1] (n2) at (-2,3) {Ontology};
        \node[draw, align=center, fill=col1] (n3) at (-3,-2.5) {Prime Number};
        \node[draw, align=center, fill=col1] (n4) at (0, 1) {Beamer};
        \node[draw, align=center, fill=col1] (n5) at (2,-2) {\LaTeX};
        \node[draw, align=center, fill=col1] (n6) at (4,2) {Python};
        \draw[-, line width=1] (n1) -- (n3);
        \draw[-, line width=1] (n1) -- (n2);
        \draw[-, line width=1] (n2) -- (n3);
        \draw[-, line width=1] (n4) -- (n5);
        \draw[-, line width=1] (n4) -- (n6);
        \draw[-, line width=1] (n2) -- (n6);
        \draw[-, line width=1] (n5) -- (n6);
      \end{pgfonlayer}
      
      \begin{pgfonlayer}{background layer}
        \foreach \n in {1,...,6}{
          \filldraw[black] ($(n\n.north west) + (.2em, -.2em)$) rectangle ($(n\n.south east) + (.2em, -.2em)$);
        };
      \end{pgfonlayer}
    \end{tikzpicture}
  \end{figure}
\end{frame}

\begin{frame}[c]{Fonts}
  \begin{columns}
    \begin{column}{.7\textwidth}
      \begin{description}
      \item[italics] \textit{The fast bulldog jumps the great happy wizard}
      \item[bold] \textbf{The fast bulldog jumps the great happy wizard}
      \item[smallcaps] \textsc{The fast bulldog jumps the great happy wizard}
      \item[roman] {\rmfamily The fast bulldog jumps the great happy wizard}
      \item[source] \texttt{The fast bulldog jumps the great happy wizard}
      \end{description}
    \end{column}
    \vrule
    \begin{column}{.3\textwidth}
    \end{column}
  \end{columns}
\end{frame}

\begin{frame}[fragile]{Typesetting Mathematics}
  \begin{block}{Gaussian Probability Density Function}
    \[
      f \left(x \mid \mu, \sigma^2 \right) = \dfrac{1}{\sqrt{2 \sigma^2 \pi}} e^{- \dfrac{(x-\mu)^2}{2\sigma^2}}
    \]
  \end{block}

\end{frame}

\end{document}
